
% ---- Begin SI numbering with S-prefix ----
\setcounter{equation}{0}
\renewcommand{\theequation}{S\arabic{equation}}

\setcounter{figure}{0}
\renewcommand{\thefigure}{S\arabic{figure}}

\setcounter{table}{0}
\renewcommand{\thetable}{S\arabic{table}}

% If you use theorem/lemma environments:
\setcounter{theorem}{0}
\renewcommand{\thetheorem}{S\arabic{theorem}}
% ---- End SI numbering setup ----
% If you use theorem/lemma environments:
\setcounter{definition}{0}
\renewcommand{\thedefinition}{S\arabic{definition}}
% ---- End SI numbering setup ----

% If you use theorem/lemma environments:
\setcounter{lemma}{0}
\renewcommand{\thelemma}{S\arabic{lemma}}
% If you use theorem/lemma environments:
\setcounter{corollary}{0}
\renewcommand{\thecorollary}{S\arabic{corollary}}


\section*{Supplementary Methods}


\subsection*{Sequence Space, Mutation Geometry and Two-Part Codes}\label{sec:si:setup}

Let $[q]^n$ denote the set of length-$n$ strings over an alphabet of size $q \ge 2$. For $x,y\in[q]^n$, let $d_{\Hamming}(x,y)$ denote the Hamming distance.
For $m\in\{0,1,\dots,n\}$, define the Hamming sphere
\cgather{
H_m(x)=\{y\in[q]^n : d_{\Hamming}(x,y)=m\}. \label{eq:si:hammingsphere}
}
with cardinality $|H_m(x)|=\binom{n}{m}(q-1)^m$, where $m$ represents the number of mutated sites in one generation.
Let $\K(\cdot)$ denote prefix-free Kolmogorov complexity with respect to a fixed universal Turing machine~\cite{LiVitanyi2019}.
If $x\in S$ and $S$ is a finite set, then the two-part code bound is as follows:
\cgather{
\K(x) \le \K(S)+\log|S|+O(1),
\qquad
\K(x\mid S)\le \log|S|+O(1). \label{eq:si:twopart}
}
which expresses the optimality condition of two-part codes~\cite{LiVitanyi2019,GacsTrompVitanyi2001}: first describe the model $S$, then describe the index of $x$ within $S$.

Next, we note that the mutation process induces a restricted model class. For fixed parent $x$ and mutation radius $m$, the Hamming sphere $H_m(x)$ is the set of sequences accessible by an $m$-mutation. Crucially, $\K(H_m(x)\mid x,m)=O(1)$, which implies
$\K(H_m(x))=\K(x)+O(\log n)$,
where the logarithmic term accounts for encoding $(n,m,q)$.
This leads to the interpretation of \emph{mutation indexing cost} as
\cgather{
c(m)=\log|H_m(x)|=\log\binom{n}{m}+m\log(q-1). \label{eq:si:cm}
}
Thus, random mutation induces the two-part code
\cgather{
\K(y)\;\approx\;\K(x)+c(m), \label{eq:si:twopartmutation}
}
for $y$ uniformly drawn from $H_m(x)$.
Finally, for a model $S\ni y$, the randomness deficiency~\cite{LiVitanyi2019,GacsTrompVitanyi2001} is
\cgather{
\Delta_y(S)=\K(S)+\log|S|-\K(y). \label{eq:si:deficiency}
}
Thus, admissible models are restricted to mutation-accessible sets $H_m(x)$.

\begin{definition}[Net structure discovery]\label{def:si:discovery}
Fix $\Delta\ge 1$. A mutant $y\in H_m(x)$ constitutes a \emph{net structure discovery} if
$\Delta_y\big(H_m(x)\big)\ge\Delta$, or equivalently, $\K(y)\le \K(x)+c(m)-\Delta$.
\end{definition}
Net structure discovery formalizes novelty as being exceptional relative to what mutation alone would typically produce.

\begin{theorem}[Universal upper bound]\label{thm:si:upper}
For any $x\in[q]^n$, any $m$, and any $\Delta\ge 1$,
\cgather{
\Pr_{y\sim \mathrm{Unif}(H_m(x))}\big[\Delta_y(H_m(x))\ge\Delta\big]
\le
O\left(\frac{2^{-\Delta}}{|H_m(x)|}\right). \label{eq:si:upper}
}
\end{theorem}

\begin{proof}
If $\Delta_y(H_m(x))\ge\Delta$, then
$\K(y)\le \K(H_m(x))+\log|H_m(x)|-\Delta$.
By the Kraft inequality for prefix-free complexity, at most $2^{T+1}$ strings satisfy $\K(y)\le T$. Substituting
$T=\K(H_m(x))+\log|H_m(x)|-\Delta$ and dividing by $|H_m(x)|$ yields the bound.
\end{proof}

\begin{lemma}[Typical tightness for incompressible parents]\label{lem:si:tight}
Let $x$ satisfy $\K(x)\ge n\log q-O(1)$. Then for $y$ drawn uniformly from $H_m(x)$,
\cgather{
\K(y\mid x,m)=\log|H_m(x)|\pm O(1), \label{eq:si:condK}
}
with probability $1-O(1/|H_m(x)|)$. Consequently, up to polylogarithmic factors,
\cgather{
\Pr\big[\Delta_y(H_m(x))\ge\Delta\big]
=
\Theta\left(\frac{2^{-\Delta}}{|H_m(x)|}\right). \label{eq:si:tight}
}
\end{lemma}

\begin{proof}
Fix $x$ and $m$. There exists a computable bijection
\cgather{
\pi_{x,m}:\{1,\dots,|H_m(x)|\}\to H_m(x) \label{eq:si:bijection}
}
such that given $(x,m,i)$ one can compute $y=\pi_{x,m}(i)$. Hence
\cgather{
\K(y\mid x,m)\le \log|H_m(x)|+O(1) \label{eq:si:condKup}
}
for all $y\in H_m(x)$. For the lower tail, define
\cgather{
A_\Delta=\{y\in H_m(x):\K(y\mid x,m)\le \log|H_m(x)|-\Delta\}. \label{eq:si:Adef}
}
Each $y\in A_\Delta$ has a prefix-free description of length at most
$\log|H_m(x)|-\Delta$, so $|A_\Delta|\le |H_m(x)|2^{-\Delta+O(1)}$. Thus
\cgather{
\Pr[y\in A_\Delta]\le 2^{-\Delta+O(1)}. \label{eq:si:Atail}
}
Since $x$ is incompressible, the two-part code through $H_m(x)$ is optimal up to $O(\log n)$ terms. Therefore
\cgather{
\Delta_y(H_m(x))
=
\log|H_m(x)|-\K(y\mid x,m)\pm O(\log n), \label{eq:si:deficiencyexpand}
}
which yields the stated asymptotic bound.
\end{proof}

\begin{corollary}[Expected discovery under binomial mutation]\label{cor:si:Phi}
Let $X\sim \mathrm{Unif}([q]^n)$ be the parent sequence.
Let $M\sim \mathrm{Binomial}(n,\mu)$ denote the number of mutated sites,
and conditional on $(X,M=m)$ let $Y\sim \mathrm{Unif}(H_m(X))$. Define the expected discovery probability
\mltlne{
\Phi(\mu)
:=
\mathbb{P}\left(\Delta_Y(H_M(X)) \ge \Delta \right)
\\=
\sum_{m=1}^n
\Pr(M=m)\,
\Pr_{y\sim H_m(X)} \left[\Delta_y(H_m(X)) \ge \Delta \right]. \label{eq:si:PhiDef}
}
Then, for $\Delta=o(\log n)$,
\cgather{
\Phi(\mu)\asymp
2^{-\Delta}
\sum_{m=1}^n
\Pr(M=m)
\frac{1}{|H_m(X)|}. \label{eq:si:PhiAsymp}
}
\end{corollary}

\begin{proof}
For $X\sim \mathrm{Unif}([q]^n)$, the standard incompressibility bound
\cgather{
\Pr\left[K(X)\le n\log q - t\right]\le q^{-t} \label{eq:si:incomp}
}
implies that $X$ satisfies the incompressibility condition of Lemma~\ref{lem:si:tight}
with overwhelming probability. On this event, Lemma~\ref{lem:si:tight} yields
\cgather{
\Pr_{y\sim H_m(X)}
\left[\Delta_y(H_m(X)) \ge \Delta \right]
\asymp
\frac{2^{-\Delta}}{|H_m(X)|}. \label{eq:si:probm}
}
The contribution of the exceptional set $\{X: K(X)\le n\log q - t\}$ is at most $q^{-t}$ and is negligible
for $t=\omega(1)$. Substituting the tight bound into the definition of $\Phi(\mu)$ gives the stated expression.
\end{proof}

Note that upon substituting the binomial mutation model and the sphere cardinality
$|H_m(X)|=\binom{n}{m}(q-1)^m$, the combinatorial factor $\binom{n}{m}$ appearing in $\Pr(M=m)$ cancels
with the identical factor in $|H_m(X)|$. Thus the multiplicity of mutation-location choices does not amplify
discovery probability: it is already accounted for by the mutation process itself. What remains is the symbol-choice
entropy $(q-1)^m$ and the exponential suppression arising from the rarity of exceptional strings within each
mutation-accessible model. We can therefore state the small-$\mu$ asymptotics of the discovery rate as follows.

\begin{lemma}[Small-$\mu$ asymptotics of discovery rate]\label{lem:si:smallmu}
For $\mu=o(1)$,
\cgather{
\sum_{m=1}^n\left(\frac{\mu}{q-1}\right)^m
=
\frac{\mu}{q-1}+O(\mu^2),
\qquad
(1-\mu)^n=e^{-n\mu+O(n\mu^2)}. \label{eq:si:smallmu}
}
Consequently,
\cgather{
\Phi(\mu)\asymp
C\,2^{-\Delta}\,\frac{\mu}{q-1}\,e^{-n\mu+O(n\mu^2)}, \label{eq:si:PhiSmallmu}
}
where $C$ absorbs polylogarithmic factors.
\end{lemma}

\begin{proof}
The first expansion is the truncated geometric series. Substituting
$n\log(1-\mu)=-n\mu+O(n\mu^2)$ into the expression for $\Phi(\mu)$ gives the claim.
\end{proof}

\begin{theorem}[Optimal mutation rate]\label{thm:si:opt}
In the regime $\mu=o(1)$ with $n\mu^2=o(1)$, the leading term $\mu e^{-n\mu}$ is maximized at $n\mu=1$.
Consequently,
\cgather{
\mu^\star=\frac{1+o(1)}{n}. \label{eq:si:muopt}
}
\end{theorem}

\begin{corollary}[Drake’s rule]\label{cor:si:drake}
Optimizing information-theoretic structure discovery under random mutation yields an inverse scaling of per-site
mutation rate with genome length.
\end{corollary}



% --- Drake's rule: 3D-ish Hamming sphere (wireframe) + selected variants ---
% Requires: \usepackage{tikz}
% Optional (recommended): \usetikzlibrary{calc}

%% \begin{figure}[t]
%% \centering
%% \begin{tikzpicture}[scale=1]
%% \def\SCALE{.8}
%% % =======================
%% % (A) Full Hamming sphere
%% % =======================
%% \node[scale=\SCALE] (A) at (0,0) {
%% \begin{tikzpicture}[ line cap=round, line join=round]

%% \def\R{2.35}        % "sphere" radius (projected)
%% \def\ry{0.58}       % vertical squash to suggest depth
%% \def\tilt{18}       % degrees: slight tilt effect via shifted longitudes

%% % Outer silhouette
%% \draw[thick] (0,0) circle (\R);

%% % Latitude rings (ellipses)
%% \foreach \s/\op in {1.00/0.55,0.86/0.35,0.72/0.25}{
%%   \draw[thin, opacity=\op] (0,0) ellipse ({\R*\s} and {\R*\ry*\s});
%% }

%% % Longitude rings (rotate ellipse by multiple angles)
%% \foreach \a/\op in {0/0.40,35/0.28,70/0.22,105/0.18,140/0.14}{
%%   \begin{scope}[rotate=\a]
%%     \draw[thin, opacity=\op] (0,0) ellipse ({\R} and {\R*\ry});
%%   \end{scope}
%% }

%% % Center genome
%% \fill (0,0) circle (2.2pt);
%% \node[font=\sffamily\small, anchor=west] at (0.12,0.05) {$x$};
%% \node[font=\sffamily\small] at (0,-0.38) {\strut parent genome};

%% % Label
%% \node[font=\sffamily\small] at (0,\R+0.35) {\strut Hamming sphere $H_m(x)$};

%% % A few faint points on surface
%% \foreach \ang in {18,62,118,165,210,268,322}{
%%   \fill[opacity=0.28] ({\R*cos(\ang)},{\R*sin(\ang)}) circle (1.5pt);
%% }

%% \end{tikzpicture}};

%% % ======================================
%% % (B) Sphere faint + selected variants
%% % ======================================
%% \node[anchor=west,scale=\SCALE] (B) at ([xshift=.5in]A.east) {
%% \begin{tikzpicture}[line cap=round, line join=round]

%% \def\R{2.35}
%% \def\ry{0.58}

%% % Faint outer silhouette
%% \draw[thick, opacity=0.18] (0,0) circle (\R);

%% % Faint latitudes
%% \foreach \s/\op in {1.00/0.14,0.86/0.10,0.72/0.08}{
%%   \draw[thin, opacity=\op] (0,0) ellipse ({\R*\s} and {\R*\ry*\s});
%% }

%% % Faint longitudes
%% \foreach \a/\op in {0/0.12,35/0.09,70/0.07,105/0.06,140/0.05}{
%%   \begin{scope}[rotate=\a]
%%     \draw[thin, opacity=\op] (0,0) ellipse ({\R} and {\R*\ry});
%%   \end{scope}
%% }

%% % Center genome (de-emphasized)
%% \fill[opacity=0.35] (0,0) circle (2.0pt);
%% \node[font=\sffamily\small, anchor=west, opacity=0.6] at (0.12,0.05) {$x$};

%% % Highlighted selected variants (on the surface)
%% \foreach \ang/\lab in {25/$y_1$,92/$y_2$,150/$y_3$,230/$y_4$,310/$y_5$}{
%%   \fill ({\R*cos(\ang)},{\R*sin(\ang)}) circle (2.2pt);
%%   \draw[thick] ({\R*cos(\ang)},{\R*sin(\ang)}) circle (3.7pt);
%%   \node[font=\sffamily\small, anchor=west] at
%%     ({\R*cos(\ang)+0.12},{\R*sin(\ang)+0.06}) {\lab};
%% }

%% % Label
%% \node[font=\sffamily\small] at (-1.5,-0.38) {\strut selected variants on $H_m(x)$};

%% \end{tikzpicture}};

%% \draw[->, >=stealth, thick, dashed] (A) -- (B);
%% \end{tikzpicture}

%% \caption{Wireframe depiction of the mutation-accessible neighborhood at fixed Hamming radius $m$. Left: the full sphere $H_m(x)$ around parent genome $x$ (wireframe to emphasize geometry). Right: the same sphere is de-emphasized and a small subset of outcomes on the sphere are highlighted as variants amplified by selection.}
%% \end{figure}



% 3D-like Hamming sphere, parameterized to fit a fixed total width (e.g., 3.0in).
% Key idea: draw each sphere in normalized coordinates (radius = 1), then set x,y units
% to the desired physical radius in the outer panel. No nested tikzpictures.

% Requires: \usepackage{tikz}
% Optional: \usetikzlibrary{calc}

%% \begin{figure}[t]
%% \centering
%% \begin{tikzpicture}[line cap=round, line join=round]

%% % -------------------------
%% % User knobs
%% % -------------------------
%% \def\TotalW{3.00in}   % total width of the whole panel (single column)
%% \def\GapW{0.18in}     % gap between left and right spheres
%% \def\ry{0.58}         % vertical squash for wireframe lat/long ellipses

%% % -------------------------
%% % Derived geometry (no touch)
%% % -------------------------
%% \pgfmathsetlengthmacro{\PanelW}{0.5*(\TotalW-\GapW)} % width allotted to each sphere panel
%% \pgfmathsetlengthmacro{\Rad}{0.5*\PanelW}            % physical radius for each sphere (in TeX length)
%% \pgfmathsetlengthmacro{\CxA}{\Rad}                   % center x of left sphere
%% \pgfmathsetlengthmacro{\CxB}{\PanelW+\GapW+\Rad}     % center x of right sphere
%% \pgfmathsetlengthmacro{\Cy}{0pt}                     % common center y

%% % Common wireframe sets in normalized (radius=1) coordinates
%% \def\LatSet{1.00/0.55,0.86/0.35,0.72/0.25}
%% \def\LonSet{0/0.40,35/0.28,70/0.22,105/0.18,140/0.14}

%% % =======================
%% % (A) Full Hamming sphere
%% % =======================
%% \begin{scope}[shift={(\CxA,\Cy)}, x=\Rad, y=\Rad]

%%   % Outer silhouette (unit circle)
%%   \draw[thick] (0,0) circle (1);

%%   % Latitudes
%%   \foreach \s/\op in \LatSet{
%%     \draw[thin, opacity=\op] (0,0) ellipse ({\s} and {\ry*\s});
%%   }

%%   % Longitudes
%%   \foreach \a/\op in \LonSet{
%%     \begin{scope}[rotate=\a]
%%       \draw[thin, opacity=\op] (0,0) ellipse (1 and \ry);
%%     \end{scope}
%%   }

%%   % Center genome
%%   \fill (0,0) circle (1.8pt);
%%   \node[font=\sffamily\small, anchor=west] at (0.06,0.03) {$x$};
%%   \node[font=\sffamily\footnotesize] at (0,-0.20) {\strut parent genome};

%%   % Label above sphere
%%   \node[font=\sffamily\small,align=center] at (0,1.4) {\strut Hamming sphere \\ $H_m(x)$};

%%   % A few faint surface points (on the outer silhouette)
%%   \foreach \ang in {18,62,118,165,210,268,322}{
%%     \fill[opacity=0.28] ({cos(\ang)},{sin(\ang)}) circle (1.4pt);
%%   }

%% \end{scope}

%% % ======================================
%% % (B) Sphere faint + selected variants
%% % ======================================
%% \begin{scope}[shift={(\CxB,\Cy)}, x=\Rad, y=\Rad]

%%   % Faint outer silhouette
%%   \draw[thick, opacity=0.1] (0,0) circle (1);

%%   % Faint latitudes
%%   \foreach \s/\op in {1.00/0.14,0.86/0.12,0.72/0.1}{
%%     \draw[thin, opacity=\op] (0,0) ellipse ({\s} and {\ry*\s});
%%   }

%%   % Faint longitudes
%%   \foreach \a/\op in {0/0.11,35/0.15,70/0.14,105/0.14,140/0.12}{
%%     \begin{scope}[rotate=\a]
%%       \draw[thin, opacity=\op] (0,0) ellipse (1 and \ry);
%%     \end{scope}
%%   }

%%   % Center genome, de-emphasized
%%   \fill[opacity=0.35] (0,0) circle (1.6pt);
%%   \node[font=\sffamily\small, anchor=west, opacity=0.6] at (0.06,0.03) {$x$};

%%   % Highlighted selected variants on the sphere
%%   \foreach \ang/\lab in {25/$y_1$,92/$y_2$,150/$y_3$,230/$y_4$,310/$y_5$}{
%%     \fill ({cos(\ang)},{sin(\ang)}) circle (1.9pt);
%%     \draw[thick] ({cos(\ang)},{sin(\ang)}) circle (3.2pt);
%%     \node[font=\sffamily\small, anchor=west] at ({cos(\ang)+0.07},{sin(\ang)+0.03}) {\lab};
%%   }

%%   \node[font=\sffamily\small,align=center] at (0,1.4) {\strut selected variants \\ on $H_m(x)$};
%%   %\node[font=\sffamily\small] at (0,1.18) {\strut Hamming sphere $H_m(x)$};

%% \end{scope}

%% % Dashed arrow between panels (in absolute TeX lengths)
%% %% \draw[->, >=stealth, thick, dashed]
%% %%   ([xshift=-30pt]\CxA+\Rad,\Cy) -- ([xshift=40pt]\CxB-\Rad,\Cy);
%% \end{tikzpicture}

%% \caption{Wireframe depiction of the mutation-accessible neighborhood at fixed Hamming radius $m$. Left: the full neighborhood $H_m(x)$ around parent genome $x$. Right: the same sphere is de-emphasized and a small subset of outcomes on the sphere are highlighted as variants amplified by selection.}
%% \end{figure}



%% \begin{figure}[t]
%% \centering
%% \begin{tikzpicture}[line cap=round, line join=round]

%% % -------------------------
%% % User knobs
%% % -------------------------
%% \def\TotalW{3.00in}   % total width of the whole panel (single column)
%% \def\GapW{0.18in}     % gap between left and right spheres
%% \def\ry{0.58}         % vertical squash for wireframe lat/long ellipses

%% % -------------------------
%% % Derived geometry (no touch)
%% % -------------------------
%% \pgfmathsetlengthmacro{\PanelW}{0.5*(\TotalW-\GapW)}
%% \pgfmathsetlengthmacro{\Rad}{0.5*\PanelW}
%% \pgfmathsetlengthmacro{\CxA}{\Rad}
%% \pgfmathsetlengthmacro{\CxB}{\PanelW+\GapW+\Rad}
%% \pgfmathsetlengthmacro{\Cy}{0pt}

%% % Common wireframe sets in normalized (radius=1) coordinates
%% \def\LatSet{1.00/0.55,0.86/0.35,0.72/0.25}
%% \def\LonSet{0/0.40,35/0.28,70/0.22,105/0.18,140/0.14}

%% % ==================================
%% % (A) Full Hamming sphere (wireframe)
%% % Suggested changes:
%% %  - radial arrow labeled m
%% %  - caption/labels emphasize "projection of a discrete shell"
%% % ==================================
%% \begin{scope}[shift={(\CxA,\Cy)}, x=\Rad, y=\Rad]

%%   % Outer silhouette (unit circle)
%%   \draw[thick] (0,0) circle (1);

%%   % Latitudes
%%   \foreach \s/\op in \LatSet{
%%     \draw[thin, opacity=\op] (0,0) ellipse ({\s} and {\ry*\s});
%%   }

%%   % Longitudes
%%   \foreach \a/\op in \LonSet{
%%     \begin{scope}[rotate=\a]
%%       \draw[thin, opacity=\op] (0,0) ellipse (1 and \ry);
%%     \end{scope}
%%   }

%%   % Center genome
%%   \fill (0,0) circle (1.8pt);
%%   \node[font=\sffamily\small, anchor=west] at (0.06,0.03) {$x$};
%%   \node[font=\sffamily\footnotesize] at (0,-0.20) {\strut parent genome};

%%   % Radial arrow indicating Hamming radius m
%%   \draw[->, >=stealth, thick] (0,0) -- ({cos(35)},{sin(35)})
%%     node[pos=0.62, font=\sffamily\small, fill=white, inner sep=1pt] {$m$};

%%   % Label above sphere (make "projection" explicit)
%%   \node[font=\sffamily\small,align=center] at (0,1.42)
%%     {\strut projection of the \\ Hamming shell $H_m(x)$};

%%   % Faint surface points (background outcomes)
%%   \foreach \ang in {18,62,118,165,210,268,322}{
%%     \fill[opacity=0.25] ({cos(\ang)},{sin(\ang)}) circle (1.4pt);
%%   }

%% \end{scope}

%% % ======================================
%% % (B) Sphere faint + selected variants
%% % Suggested changes:
%% %  - selected points are clustered (non-uniform) to avoid "uniform random"
%% %  - optional faint "selected cap" patch to indicate structure
%% %  - keep sphere de-emphasized
%% % ======================================
%% \begin{scope}[shift={(\CxB,\Cy)}, x=\Rad, y=\Rad]

%%   % Faint outer silhouette
%%   \draw[thick, opacity=0.10] (0,0) circle (1);

%%   % Faint latitudes
%%   \foreach \s/\op in {1.00/0.10,0.86/0.09,0.72/0.08}{
%%     \draw[thin, opacity=\op] (0,0) ellipse ({\s} and {\ry*\s});
%%   }

%%   % Faint longitudes
%%   \foreach \a/\op in {0/0.09,35/0.08,70/0.07,105/0.06,140/0.05}{
%%     \begin{scope}[rotate=\a]
%%       \draw[thin, opacity=\op] (0,0) ellipse (1 and \ry);
%%     \end{scope}
%%   }

%%   % Center genome, de-emphasized
%%   \fill[opacity=0.25] (0,0) circle (1.6pt);
%%   \node[font=\sffamily\small, anchor=west, opacity=0.5] at (0.06,0.03) {$x$};

%%   % Optional: faint "cap" patch where selected variants tend to lie
%%   % (drawn as a short arc on the outer silhouette)
%%   \draw[very thick, opacity=0.20] ( {cos(18)}, {sin(18)} )
%%     arc[start angle=18, end angle=70, radius=1];

%%   % Highlighted selected variants (clustered on a patch of the shell)
%%   % (angles concentrated to indicate structured selection)
%%   \foreach \ang/\lab in {14/$y_1$,32/$y_2$,41/$y_3$,52/$y_4$,63/$y_5$}{
%%     \fill ({cos(\ang)},{sin(\ang)}) circle (1.9pt);
%%     \draw[thick] ({cos(\ang)},{sin(\ang)}) circle (3.2pt);
%%     \node[font=\sffamily\small, anchor=west] at ({cos(\ang)+0.07},{sin(\ang)+0.03}) {\lab};
%%   }

%%   \node[font=\sffamily\small,align=center] at (0,1.42)
%%     {\strut selection retains a \\ sparse subset on $H_m(x)$};

%% \end{scope}

%% \end{tikzpicture}

%% \caption{Low-dimensional schematic of the mutation-accessible neighborhood at fixed Hamming radius $m$. Left: the discrete shell $H_m(x)$ around parent genome $x$, shown as a wireframe sphere for geometric intuition. Right: the shell is de-emphasized and only a small, structured subset of outcomes on $H_m(x)$ are highlighted as variants amplified by selection.}
%% \end{figure}


%% % Three-stage minimalist schematic:
%% %   (1) single parent dot x
%% %   (2) expansion: disk uniformly filled with faint dots
%% %   (3) selection: same disk faded, with 2-3 highlighted dots (colored)
%% %
%% % Requires: \usepackage{tikz}
%% % Optional: \usetikzlibrary{calc,arrows.meta}

%% \begin{figure}[t]
%% \centering
%% \begin{tikzpicture}[line cap=round, line join=round, >=stealth]

%% % -------------------------
%% % User knobs
%% % -------------------------
%% \def\TotalW{3.00in}     % overall width target
%% \def\R{0.42in}          % disk radius
%% \def\Gap{0.18in}        % gap between stages
%% \def\N{85}              % number of background dots in disk

%% % Stage x positions
%% \pgfmathsetlengthmacro{\xA}{0.10in}
%% \pgfmathsetlengthmacro{\xB}{\xA + 0.55in + \Gap}
%% \pgfmathsetlengthmacro{\xC}{\xB + 2*\R + 0.55in + \Gap}
%% \pgfmathsetlengthmacro{\yC}{0in}

%% % -------------------------
%% % (1) Parent x
%% % -------------------------
%% \fill (\xA,\yC) circle (2.2pt);
%% \node[font=\sffamily\small, anchor=west] at (\xA+0.08in,\yC+0.01in) {$x$};

%% \draw[->, thick] (\xA+0.32in,\yC) -- (\xB-\R-0.10in,\yC)
%%   node[midway, font=\sffamily\footnotesize, yshift=8pt] {mutation};

%% % -------------------------
%% % Helper: deterministic "uniform-ish" fill in a disk (no RNG)
%% % -------------------------
%% % We place points using a sunflower (Fermat spiral) pattern:
%% % r_k = R * sqrt(k/N), theta_k = k * golden_angle
%% \pgfmathsetmacro{\gold}{137.50776405} % degrees

%% % -------------------------
%% % (2) Expansion disk
%% % -------------------------
%% \begin{scope}[shift={(\xB,\yC)}]

%%   % boundary (optional; keep light)
%%   \draw[thin, opacity=0.30] (0,0) circle (\R);

%%   % filled dots
%%   \foreach \k in {1,...,\N}{
%%     \pgfmathsetmacro{\t}{\k*\gold}
%%     \pgfmathsetmacro{\rr}{sqrt(\k/\N)}
%%     \fill[opacity=0.28] ({\rr*\R*cos(\t)},{\rr*\R*sin(\t)}) circle (1.15pt);
%%   }

%% \end{scope}

%% \draw[->, thick] (\xB+\R+0.10in,\yC) -- (\xC-\R-0.10in,\yC)
%%   node[midway, font=\sffamily\footnotesize, yshift=8pt] {selection};

%% % -------------------------
%% % (3) Selection disk: same, faded + a few highlighted points
%% % -------------------------
%% \begin{scope}[shift={(\xC,\yC)}]

%%   % boundary (faded)
%%   \draw[thin, opacity=0.12] (0,0) circle (\R);

%%   % same filled dots, but more faded
%%   \foreach \k in {1,...,\N}{
%%     \pgfmathsetmacro{\t}{\k*\gold}
%%     \pgfmathsetmacro{\rr}{sqrt(\k/\N)}
%%     \fill[opacity=0.08] ({\rr*\R*cos(\t)},{\rr*\R*sin(\t)}) circle (1.15pt);
%%   }

%%   % highlighted retained variants (same size, different color)
%%   % pick a few indices so they lie inside the disk
%%   \foreach \k/\lab in {12/$y_1$,37/$y_2$,71/$y_3$}{
%%     \pgfmathsetmacro{\t}{\k*\gold}
%%     \pgfmathsetmacro{\rr}{sqrt(\k/\N)}
%%     \fill[red] ({\rr*\R*cos(\t)},{\rr*\R*sin(\t)}) circle (1.15pt);
%%     % optional tiny labels (comment out if you want ultra-minimal)
%%     %\node[font=\sffamily\footnotesize, anchor=west] at
%%     %  ({\rr*\R*cos(\t)+0.06in},{\rr*\R*sin(\t)+0.02in}) {\lab};
%%   }

%% \end{scope}

%% \end{tikzpicture}

%% \caption{Two-step schematic: mutation expands a parent genome $x$ into many variants (uniform fill), and selection retains only a sparse subset (highlighted) from the same candidate set.}
%% \end{figure}












%% % Three-stage schematic:
%% %   (1) single parent x
%% %   (2) expansion: disk uniformly filled with dots (including central x)
%% %   (3) selection: same disk faded, few highlighted dots, central x retained
%% %
%% % Requires: \usepackage{tikz}
%% % Optional: \usetikzlibrary{calc,arrows.meta}

%% \begin{figure}[t]
%% \centering
%% \begin{tikzpicture}[line cap=round, line join=round, >=stealth]

%% % -------------------------
%% % User knobs
%% % -------------------------
%% \def\TotalW{3.00in}
%% \def\R{0.42in}          % disk radius
%% \def\Gap{0.18in}
%% \def\N{85}              % background dots

%% % Stage x positions
%% \pgfmathsetlengthmacro{\xA}{0.10in}
%% \pgfmathsetlengthmacro{\xB}{\xA + 0.55in + \Gap}
%% \pgfmathsetlengthmacro{\xC}{\xB + 2*\R + 0.55in + \Gap}
%% \pgfmathsetlengthmacro{\yC}{0in}

%% % Sunflower angle (deterministic fill)
%% \pgfmathsetmacro{\gold}{137.50776405}

%% % -------------------------
%% % (1) Parent x
%% % -------------------------
%% \fill (\xA,\yC) circle (2.2pt);
%% \node[font=\sffamily\small, anchor=west] at (\xA,\yC-0.1in) {$x$};

%% \draw[->, thick] (\xA+0.02in,\yC) -- (\xB-\R-0.0in,\yC)
%%   node[midway, above, font=\sffamily\footnotesize, yshift=2pt, xshift=-6pt] {mutation};

%% % -------------------------
%% % (2) Expansion disk
%% % -------------------------
%% \begin{scope}[shift={(\xB,\yC)}]

%%   % boundary (light)
%%   \draw[thin, opacity=0.30] (0,0) circle (\R);

%%   % central parent x still present
%%   \fill (0,0) circle (2.0pt);
%%   \node[font=\sffamily\footnotesize, anchor=west] at (0.06in,0.01in) {$x$};

%%   % filled dots (mutation outcomes)
%%   \foreach \k in {1,...,\N}{
%%     \pgfmathsetmacro{\t}{\k*\gold}
%%     \pgfmathsetmacro{\rr}{sqrt(\k/\N)}
%%     \fill[opacity=0.28]
%%       ({\rr*\R*cos(\t)},{\rr*\R*sin(\t)}) circle (1.1pt);
%%   }

%% \end{scope}

%% \draw[->, thick] (\xB+\R+0.10in,\yC) -- (\xC-\R-0.10in,\yC)
%%   node[midway, font=\sffamily\footnotesize, yshift=8pt] {selection};

%% % -------------------------
%% % (3) Selection disk
%% % -------------------------
%% \begin{scope}[shift={(\xC,\yC)}]

%%   % boundary (faded)
%%   \draw[thin, opacity=0.12] (0,0) circle (\R);

%%   % central x retained (same size, black)
%%   \fill (0,0) circle (2.0pt);
%%   \node[font=\sffamily\footnotesize, anchor=west] at (0.06in,0.01in) {$x$};

%%   % background candidates (faded)
%%   \foreach \k in {1,...,\N}{
%%     \pgfmathsetmacro{\t}{\k*\gold}
%%     \pgfmathsetmacro{\rr}{sqrt(\k/\N)}
%%     \fill[opacity=0.08]
%%       ({\rr*\R*cos(\t)},{\rr*\R*sin(\t)}) circle (1.1pt);
%%   }

%%   % highlighted retained variants (same size, different color)
%%   \foreach \k/\lab in {12/$y_1$,37/$y_2$,71/$y_3$}{
%%     \pgfmathsetmacro{\t}{\k*\gold}
%%     \pgfmathsetmacro{\rr}{sqrt(\k/\N)}
%%     \fill[red]
%%       ({\rr*\R*cos(\t)},{\rr*\R*sin(\t)}) circle (1.1pt);
%%   }

%% \end{scope}

%% % -------------------------
%% % Operator equation
%% % -------------------------
%% \node[font=\sffamily\footnotesize]
%%   at (0.5*\TotalW,-0.65in)
%%   {$x \xrightarrow{\ \mathrm{mutation}\ } H_m(x)
%%     \xrightarrow{\ \mathrm{selection}\ } \{x, y_1,\dots,y_k\}$};

%% \end{tikzpicture}

%% \caption{Two-step evolutionary process: mutation expands a parent genome $x$ into many nearby variants (uniform fill), and selection retains only a sparse subset while the parent lineage persists.}
%% \end{figure}
