
% ---- Begin SI numbering with S-prefix ----
\setcounter{equation}{0}
\renewcommand{\theequation}{S\arabic{equation}}

\setcounter{figure}{0}
\renewcommand{\thefigure}{S\arabic{figure}}

\setcounter{table}{0}
\renewcommand{\thetable}{S\arabic{table}}

% If you use theorem/lemma environments:
\setcounter{theorem}{0}
\renewcommand{\thetheorem}{S\arabic{theorem}}
% ---- End SI numbering setup ----
% If you use theorem/lemma environments:
\setcounter{definition}{0}
\renewcommand{\thedefinition}{S\arabic{definition}}
% ---- End SI numbering setup ----

% If you use theorem/lemma environments:
\setcounter{lemma}{0}
\renewcommand{\thelemma}{S\arabic{lemma}}
% If you use theorem/lemma environments:
\setcounter{corollary}{0}
\renewcommand{\thecorollary}{S\arabic{corollary}}


\section*{Supplementary Methods}


\subsection*{Sequence Space, Mutation Geometry and Two-Part Codes}\label{sec:si:setup}

Let $[q]^n$ denote the set of length-$n$ strings over an alphabet of size $q \ge 2$. For $x,y\in[q]^n$, let $d_{\Hamming}(x,y)$ denote the Hamming distance.
For $m\in\{0,1,\dots,n\}$, define the Hamming sphere
\cgather{
H_m(x)=\{y\in[q]^n : d_{\Hamming}(x,y)=m\}. \label{eq:si:hammingsphere}
}
with cardinality $|H_m(x)|=\binom{n}{m}(q-1)^m$, where $m$ represents the number of mutated sites in one generation.
Let $\K(\cdot)$ denote prefix-free Kolmogorov complexity with respect to a fixed universal Turing machine~\cite{LiVitanyi2019}.
If $x\in S$ and $S$ is a finite set, then the two-part code bound is as follows:
\cgather{
\K(x) \le \K(S)+\log|S|+O(1),
\qquad
\K(x\mid S)\le \log|S|+O(1). \label{eq:si:twopart}
}
which expresses the optimality condition of two-part codes~\cite{LiVitanyi2019,GacsTrompVitanyi2001}: first describe the model $S$, then describe the index of $x$ within $S$.

Next, we note that the mutation process induces a restricted model class. For fixed parent $x$ and mutation radius $m$, the Hamming sphere $H_m(x)$ is the set of sequences accessible by an $m$-mutation. Crucially, $\K(H_m(x)\mid x,m)=O(1)$, which implies
$\K(H_m(x))=\K(x)+O(\log n)$,
where the logarithmic term accounts for encoding $(n,m,q)$.
This leads to the interpretation of \emph{mutation indexing cost} as
\cgather{
c(m)=\log|H_m(x)|=\log\binom{n}{m}+m\log(q-1). \label{eq:si:cm}
}
Thus, random mutation induces the two-part code
\cgather{
\K(y)\;\approx\;\K(x)+c(m), \label{eq:si:twopartmutation}
}
for $y$ uniformly drawn from $H_m(x)$.
Finally, for a model $S\ni y$, the randomness deficiency~\cite{LiVitanyi2019,GacsTrompVitanyi2001} is
\cgather{
\Delta_y(S)=\K(S)+\log|S|-\K(y). \label{eq:si:deficiency}
}
Thus, admissible models are restricted to mutation-accessible sets $H_m(x)$.

\begin{definition}[Net structure discovery]\label{def:si:discovery}
Fix $\Delta\ge 1$. A mutant $y\in H_m(x)$ constitutes a \emph{net structure discovery} if
$\Delta_y\big(H_m(x)\big)\ge\Delta$, or equivalently, $\K(y)\le \K(x)+c(m)-\Delta$.
\end{definition}
Net structure discovery formalizes novelty as being exceptional relative to what mutation alone would typically produce.

\begin{theorem}[Universal upper bound]\label{thm:si:upper}
For any $x\in[q]^n$, any $m$, and any $\Delta\ge 1$,
\cgather{
\Pr_{y\sim \mathrm{Unif}(H_m(x))}\big[\Delta_y(H_m(x))\ge\Delta\big]
\le
O\left(\frac{2^{-\Delta}}{|H_m(x)|}\right). \label{eq:si:upper}
}
\end{theorem}

\begin{proof}
If $\Delta_y(H_m(x))\ge\Delta$, then
$\K(y)\le \K(H_m(x))+\log|H_m(x)|-\Delta$.
By the Kraft inequality for prefix-free complexity, at most $2^{T+1}$ strings satisfy $\K(y)\le T$. Substituting
$T=\K(H_m(x))+\log|H_m(x)|-\Delta$ and dividing by $|H_m(x)|$ yields the bound.
\end{proof}

\begin{lemma}[Typical tightness for incompressible parents]\label{lem:si:tight}
Let $x$ satisfy $\K(x)\ge n\log q-O(1)$. Then for $y$ drawn uniformly from $H_m(x)$,
\cgather{
\K(y\mid x,m)=\log|H_m(x)|\pm O(1), \label{eq:si:condK}
}
with probability $1-O(1/|H_m(x)|)$. Consequently, up to polylogarithmic factors,
\cgather{
\Pr\big[\Delta_y(H_m(x))\ge\Delta\big]
=
\Theta\left(\frac{2^{-\Delta}}{|H_m(x)|}\right). \label{eq:si:tight}
}
\end{lemma}

\begin{proof}
Fix $x$ and $m$. There exists a computable bijection
\cgather{
\pi_{x,m}:\{1,\dots,|H_m(x)|\}\to H_m(x) \label{eq:si:bijection}
}
such that given $(x,m,i)$ one can compute $y=\pi_{x,m}(i)$. Hence
\cgather{
\K(y\mid x,m)\le \log|H_m(x)|+O(1) \label{eq:si:condKup}
}
for all $y\in H_m(x)$. For the lower tail, define
\cgather{
A_\Delta=\{y\in H_m(x):\K(y\mid x,m)\le \log|H_m(x)|-\Delta\}. \label{eq:si:Adef}
}
Each $y\in A_\Delta$ has a prefix-free description of length at most
$\log|H_m(x)|-\Delta$, so $|A_\Delta|\le |H_m(x)|2^{-\Delta+O(1)}$. Thus
\cgather{
\Pr[y\in A_\Delta]\le 2^{-\Delta+O(1)}. \label{eq:si:Atail}
}
Since $x$ is incompressible, the two-part code through $H_m(x)$ is optimal up to $O(\log n)$ terms. Therefore
\cgather{
\Delta_y(H_m(x))
=
\log|H_m(x)|-\K(y\mid x,m)\pm O(\log n), \label{eq:si:deficiencyexpand}
}
which yields the stated asymptotic bound.
\end{proof}

\begin{corollary}[Expected discovery under binomial mutation]\label{cor:si:Phi}
Let $X\sim \mathrm{Unif}([q]^n)$ be the parent sequence.
Let $M\sim \mathrm{Binomial}(n,\mu)$ denote the number of mutated sites,
and conditional on $(X,M=m)$ let $Y\sim \mathrm{Unif}(H_m(X))$. Define the expected discovery probability
\mltlne{
\Phi(\mu)
:=
\mathbb{P}\left(\Delta_Y(H_M(X)) \ge \Delta \right)
\\=
\sum_{m=1}^n
\Pr(M=m)\,
\Pr_{y\sim H_m(X)} \left[\Delta_y(H_m(X)) \ge \Delta \right]. \label{eq:si:PhiDef}
}
Then, for $\Delta=o(\log n)$,
\cgather{
\Phi(\mu)\asymp
2^{-\Delta}
\sum_{m=1}^n
\Pr(M=m)
\frac{1}{|H_m(X)|}. \label{eq:si:PhiAsymp}
}
\end{corollary}

\begin{proof}
For $X\sim \mathrm{Unif}([q]^n)$, the standard incompressibility bound
\cgather{
\Pr\left[K(X)\le n\log q - t\right]\le q^{-t} \label{eq:si:incomp}
}
implies that $X$ satisfies the incompressibility condition of Lemma~\ref{lem:si:tight}
with overwhelming probability. On this event, Lemma~\ref{lem:si:tight} yields
\cgather{
\Pr_{y\sim H_m(X)}
\left[\Delta_y(H_m(X)) \ge \Delta \right]
\asymp
\frac{2^{-\Delta}}{|H_m(X)|}. \label{eq:si:probm}
}
The contribution of the exceptional set $\{X: K(X)\le n\log q - t\}$ is at most $q^{-t}$ and is negligible
for $t=\omega(1)$. Substituting the tight bound into the definition of $\Phi(\mu)$ gives the stated expression.
\end{proof}

Note that upon substituting the binomial mutation model and the sphere cardinality
$|H_m(X)|=\binom{n}{m}(q-1)^m$, the combinatorial factor $\binom{n}{m}$ appearing in $\Pr(M=m)$ cancels
with the identical factor in $|H_m(X)|$. Thus the multiplicity of mutation-location choices does not amplify
discovery probability: it is already accounted for by the mutation process itself. What remains is the symbol-choice
entropy $(q-1)^m$ and the exponential suppression arising from the rarity of exceptional strings within each
mutation-accessible model. We can therefore state the small-$\mu$ asymptotics of the discovery rate as follows.


\paragraph{Remark on mutation-count distributions.}
The Binomial model arises directly from independent per-site mutation. The Poisson($T$) form used for intuition is the classical large-$n$, small-$\mu$ limit with $T=n\mu$ fixed. The exponential factor in the discovery rate originates from the probability of zero additional perturbations under this independent model. More generally, replacing the Binomial kernel by another thin-tailed mutation-count distribution replaces $(1-\mu)^n$ by the corresponding small-step mass of that kernel; the existence of an interior optimum at $T=O(1)$ persists under such replacements, although its precise location may shift.




\begin{lemma}[Small-$\mu$ asymptotics of discovery rate]\label{lem:si:smallmu}
For $\mu=o(1)$,
\cgather{
\sum_{m=1}^n\left(\frac{\mu}{q-1}\right)^m
=
\frac{\mu}{q-1}+O(\mu^2),
\qquad
(1-\mu)^n=e^{-n\mu+O(n\mu^2)}. \label{eq:si:smallmu}
}
Consequently,
\cgather{
\Phi(\mu)\asymp
C\,2^{-\Delta}\,\frac{\mu}{q-1}\,e^{-n\mu+O(n\mu^2)}, \label{eq:si:PhiSmallmu}
}
where $C$ absorbs polylogarithmic factors.
\end{lemma}

\begin{proof}
The first expansion is the truncated geometric series. Substituting
$n\log(1-\mu)=-n\mu+O(n\mu^2)$ into the expression for $\Phi(\mu)$ gives the claim.
\end{proof}

\begin{theorem}[Optimal mutation rate]\label{thm:si:opt}
In the regime $\mu=o(1)$ with $n\mu^2=o(1)$, the leading term $\mu e^{-n\mu}$ is maximized at $n\mu=1$.
Consequently,
\cgather{
\mu^\star=\frac{1+o(1)}{n}. \label{eq:si:muopt}
}
\end{theorem}

\begin{corollary}[Drake’s rule]\label{cor:si:drake}
Optimizing information-theoretic structure discovery under random mutation yields an inverse scaling of per-site
mutation rate with genome length.
\end{corollary}

\paragraph{Robustness.}
The maximizer $n\mu=1$ is obtained under the regime $\mu=o(1)$ with $n\mu^2=o(1)$, corresponding to independent, local perturbations. For mutation kernels exhibiting strong overdispersion or correlated multi-site bursts, the detailed maximizing value can deviate, but the emergence of an interior optimum driven by the competition between supply ($\propto T$) and erosion ($\propto e^{-T}$) remains structurally intact for thin-tailed kernels.




\paragraph{Genome Dataset Description and Compression-Based Complexity Estimation.}
For the genome-compressibility analysis (Fig.~1a), we analyzed complete genomes from four organisms: H3N2 influenza A (2010--2025), SARS-CoV-2 (hCoV-19; 2020--2025), mpox virus (2022--2026), and \textit{Escherichia coli} (2021--2026). All viral genomes (H3N2, SARS-CoV-2, mpox) were obtained from the GISAID database, and \textit{E. coli} genomes were obtained from the NCBI GenBank/RefSeq genomes repository. In total, we analyzed 54{,}904 H3N2 genomes across 16 annual bins (2010--2025), 4{,}986 SARS-CoV-2 genomes across 6 annual bins (2020--2025), 3{,}429 mpox genomes across 5 annual bins (2022--2026), and 8{,}785 \textit{E. coli} genomes across 6 annual bins (2021--2026). Sequences were grouped by collection year and filtered to retain high-coverage, near-complete assemblies; sequences with substantial ambiguity (extended runs of non-ACGT symbols), obvious truncation, or pervasive gaps were excluded prior to analysis. All retained genomes were processed through an identical compression pipeline to ensure cross-organism comparability.

To estimate compressibility we computed a reference-free ``bits per base'' statistic from lossless compression. Each genome sequence was mapped to an ASCII byte string over $\{A,C,G,T\}$ after the filtering step and then compressed using \texttt{zlib} with a fixed compression level and identical settings across all organisms and years. Let $C_{\mathrm{real}}(x)$ denote the compressed length in bits of a genome $x$ of length $n$. We first computed the naive compressed bits per base $b_{\mathrm{raw}}(x)=C_{\mathrm{real}}(x)/n$. Because lossless compressors incur finite overhead and windowing effects that can bias short sequences, we applied a length-matched overhead correction: for each distinct length $n$ observed within an organism-year bin, we generated a set of IID uniform random DNA sequences of length $n$, compressed them with the same \texttt{zlib} settings, and estimated the overhead term
\[
\widehat{O}(n)=\mathrm{median}\big(C_{\mathrm{rand}}(n)\big)-2n,
\]
where $2n$ is the Shannon limit (in bits) for IID uniform DNA over a four-letter alphabet. The corrected bits-per-base value reported in Fig.~1a was then
\[
b_{\mathrm{corr}}(x)=\frac{C_{\mathrm{real}}(x)-\widehat{O}(n)}{n},
\]
with $b_{\mathrm{corr}}(x)$ truncated to the interval $[0,2]$ for interpretability. Annual means were computed by averaging $b_{\mathrm{corr}}(x)$ over all genomes in the corresponding organism-year bin. Uncertainty bands in Fig.~1a correspond to 95\% confidence intervals computed from the within-bin replicate distribution using the same procedure for all organisms and years.



\paragraph*{Generation of Figure 1b: Toy Evolution Under Mutation With and Without a Viability Filter.}

Panel b illustrates a controlled simulation on sequence space $[4]^n$ designed to contrast mutation-driven entropy increase with mutation constrained by a simple structural viability criterion. We fix an alphabet of size $q=4$ and a genome length $n$ (as specified in the accompanying code). A single ``structured template'' sequence $x^\star \in [4]^n$ is sampled once and held fixed throughout the experiment. Independent replicate lineages are initialized at this template and evolved forward for a fixed number of generations.

At each generation, offspring are generated from the current sequence by independent per-site mutation with probability $\mu$. Conditional on mutation at a site, the new symbol is chosen uniformly from the remaining $q-1$ alternatives. This defines the mutation-only dynamics. For the mutation-plus-filter condition, the same mutation step is applied, but the offspring is accepted only if its Hamming distance from the template $x^\star$ does not exceed a fixed threshold $r$. If the proposed offspring violates this constraint, it is rejected and the parent sequence is retained for that generation.

For each lineage and generation, we compute a reference-free complexity proxy defined as the zlib-compressed length (in bits) divided by genome length, yielding a bits-per-symbol statistic. Compression is applied directly to the raw symbolic sequence without alignment or model fitting. Multiple independent replicate lineages are simulated under each regime, and the plotted trajectories represent the mean across replicates as a function of generation.

Under mutation alone, trajectories approach the maximal entropy regime for a 4-letter alphabet (near $2$ bits per symbol), reflecting convergence toward statistical typicality under repeated local perturbation. Under mutation with the viability filter, trajectories remain bounded away from this maximum, as the Hamming constraint prevents diffusion into the exponentially large typical set. The panel therefore visualizes the supply--erosion tension discussed in the main text: unconstrained mutation drives sequences toward maximal entropy, whereas even a simple structural constraint maintains bounded complexity over time.
\paragraph*{Generation of Figure 1c,d: Simulation and theory for the discovery rate under independent per-site mutation.}

Panels c--d plot the expected discovery rate $\Phi(\mu)$ under the independent per-site mutation model, together with Monte Carlo (MC) estimates for selected genome lengths. We fix alphabet size $q$ (taken as $q=4$ in the simulations) and a novelty margin $\Delta$ (as specified in the code). For each genome length $n$, mutation is modeled as follows: the number of mutated sites $M$ is distributed as $M\sim\mathrm{Binomial}(n,\mu)$, and conditional on $M=m$ and a parent sequence $X\in[q]^n$, an offspring $Y$ is drawn uniformly from the Hamming sphere $H_m(X)$, implemented by choosing $m$ distinct sites uniformly without replacement and replacing each selected symbol by a uniformly chosen alternative from the remaining $q-1$ symbols. The discovery event is defined as $\Delta_Y(H_M(X))\ge \Delta$, i.e., positive randomness deficiency at least $\Delta$ relative to the mutation-induced model class $H_m(X)$.

\emph{Theory curves.} The theoretical $\Phi(\mu)$ plotted as solid lines in panel c are computed from the closed-form expression obtained by substituting the typical-case bound into the binomial mixture (Eq.~(S15) in the SI), yielding
\[
\Phi(\mu)\;\approx\; C\,2^{-\Delta}(1-\mu)^n\sum_{m=1}^n\Big(\frac{\mu}{q-1}\Big)^m,
\]
where $C$ absorbs polylogarithmic factors (set to $C=1$ for plotting), and the geometric sum is evaluated exactly up to $m=n$. Panel c displays $\Phi(\mu)$ as a function of the per-site rate $\mu$ for multiple $n$, showing the interior maximizer shifting as $\mu^\star\propto 1/n$.

\emph{Monte Carlo estimates.} For MC points (shown where available), we estimate $\Phi(\mu)$ directly by sampling $R$ independent replicates at each $(n,\mu)$. In each replicate we sample a parent $X\sim\mathrm{Unif}([q]^n)$, sample $M\sim\mathrm{Binomial}(n,\mu)$, and generate $Y$ by applying an $m$-site mutation as above. We then evaluate the discovery indicator $\mathbf{1}\{\Delta_Y(H_M(X))\ge \Delta\}$ using the SI criterion $\Delta_Y(H_m(X)) = K(H_m(X))+\log|H_m(X)|-K(Y)$ with $K(H_m(X))=K(X)+O(\log n)$ and with $K(\cdot)$ operationalized by the same code-length surrogate used throughout the manuscript (as implemented in the provided code). Averaging the indicator over replicates yields the MC estimate $\widehat{\Phi}(\mu)$.

\emph{Collapse in mutation temperature.} Panel d re-parameterizes the horizontal axis by the per-genome mutation intensity (``mutation temperature'') $T=n\mu=E[M]$. The same theoretical curves are replotted as $\Phi(T/n)$ versus $T$ for each $n$, producing a collapse onto a common profile with a maximum near $T\simeq 1$. The dashed guide curve shows the small-$\mu$ asymptotic shape $\Phi(\mu)\propto \mu e^{-n\mu}$, i.e., $\Phi(T/n)\propto T e^{-T}$ up to a multiplicative constant, derived by expanding $(1-\mu)^n\approx e^{-n\mu}$ and truncating the geometric sum to leading order in $\mu$ (SI Lemma~S2). This demonstrates that the optimal regime corresponds to an $O(1)$ expected number of mutations per genome per generation and yields the scaling $\mu^\star\approx 1/n$.
