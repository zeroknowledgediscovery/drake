
% ---- Begin SI numbering with S-prefix ----
\setcounter{equation}{0}
\renewcommand{\theequation}{S\arabic{equation}}

\setcounter{figure}{0}
\renewcommand{\thefigure}{S\arabic{figure}}

\setcounter{table}{0}
\renewcommand{\thetable}{S\arabic{table}}

% If you use theorem/lemma environments:
\setcounter{theorem}{0}
\renewcommand{\thetheorem}{S\arabic{theorem}}
% ---- End SI numbering setup ----
% If you use theorem/lemma environments:
\setcounter{definition}{0}
\renewcommand{\thedefinition}{S\arabic{definition}}
% ---- End SI numbering setup ----

% If you use theorem/lemma environments:
\setcounter{lemma}{0}
\renewcommand{\thelemma}{S\arabic{lemma}}
% If you use theorem/lemma environments:
\setcounter{corollary}{0}
\renewcommand{\thecorollary}{S\arabic{corollary}}


\section*{Supplementary Methods}


\subsection*{Sequence Space, Mutation Geometry and Two-Part Codes}\label{sec:si:setup}

Let $[q]^n$ denote the set of length-$n$ strings over an alphabet of size $q \ge 2$. For $x,y\in[q]^n$, let $d_{\Hamming}(x,y)$ denote the Hamming distance.
For $m\in\{0,1,\dots,n\}$, define the Hamming sphere
\cgather{
H_m(x)=\{y\in[q]^n : d_{\Hamming}(x,y)=m\}. \label{eq:si:hammingsphere}
}
with cardinality $|H_m(x)|=\binom{n}{m}(q-1)^m$, where $m$ represents the number of mutated sites in one generation.
Let $\K(\cdot)$ denote prefix-free Kolmogorov complexity with respect to a fixed universal Turing machine~\cite{LiVitanyi2019}.
If $x\in S$ and $S$ is a finite set, then the two-part code bound is as follows:
\cgather{
\K(x) \le \K(S)+\log|S|+O(1),
\qquad
\K(x\mid S)\le \log|S|+O(1). \label{eq:si:twopart}
}
which expresses the optimality condition of two-part codes~\cite{LiVitanyi2019,GacsTrompVitanyi2001}: first describe the model $S$, then describe the index of $x$ within $S$.

Next, we note that the mutation process induces a restricted model class. For fixed parent $x$ and mutation radius $m$, the Hamming sphere $H_m(x)$ is the set of sequences accessible by an $m$-mutation. Crucially, $\K(H_m(x)\mid x,m)=O(1)$, which implies
$\K(H_m(x))=\K(x)+O(\log n)$,
where the logarithmic term accounts for encoding $(n,m,q)$.
This leads to the interpretation of \emph{mutation indexing cost} as
\cgather{
c(m)=\log|H_m(x)|=\log\binom{n}{m}+m\log(q-1). \label{eq:si:cm}
}
Thus, random mutation induces the two-part code
\cgather{
\K(y)\;\approx\;\K(x)+c(m), \label{eq:si:twopartmutation}
}
for $y$ uniformly drawn from $H_m(x)$.
Finally, for a model $S\ni y$, the randomness deficiency~\cite{LiVitanyi2019,GacsTrompVitanyi2001} is
\cgather{
\Delta_y(S)=\K(S)+\log|S|-\K(y). \label{eq:si:deficiency}
}
Thus, admissible models are restricted to mutation-accessible sets $H_m(x)$.

\begin{definition}[Net structure discovery]\label{def:si:discovery}
Fix $\Delta\ge 1$. A mutant $y\in H_m(x)$ constitutes a \emph{net structure discovery} if
$\Delta_y\big(H_m(x)\big)\ge\Delta$, or equivalently, $\K(y)\le \K(x)+c(m)-\Delta$.
\end{definition}
Net structure discovery formalizes novelty as being exceptional relative to what mutation alone would typically produce.

\begin{theorem}[Universal upper bound]\label{thm:si:upper}
For any $x\in[q]^n$, any $m$, and any $\Delta\ge 1$,
\cgather{
\Pr_{y\sim \mathrm{Unif}(H_m(x))}\big[\Delta_y(H_m(x))\ge\Delta\big]
\le
O\left(\frac{2^{-\Delta}}{|H_m(x)|}\right). \label{eq:si:upper}
}
\end{theorem}

\begin{proof}
If $\Delta_y(H_m(x))\ge\Delta$, then
$\K(y)\le \K(H_m(x))+\log|H_m(x)|-\Delta$.
By the Kraft inequality for prefix-free complexity, at most $2^{T+1}$ strings satisfy $\K(y)\le T$. Substituting
$T=\K(H_m(x))+\log|H_m(x)|-\Delta$ and dividing by $|H_m(x)|$ yields the bound.
\end{proof}

\begin{lemma}[Typical tightness for incompressible parents]\label{lem:si:tight}
Let $x$ satisfy $\K(x)\ge n\log q-O(1)$. Then for $y$ drawn uniformly from $H_m(x)$,
\cgather{
\K(y\mid x,m)=\log|H_m(x)|\pm O(1), \label{eq:si:condK}
}
with probability $1-O(1/|H_m(x)|)$. Consequently, up to polylogarithmic factors,
\cgather{
\Pr\big[\Delta_y(H_m(x))\ge\Delta\big]
=
\Theta\left(\frac{2^{-\Delta}}{|H_m(x)|}\right). \label{eq:si:tight}
}
\end{lemma}

\begin{proof}
Fix $x$ and $m$. There exists a computable bijection
\cgather{
\pi_{x,m}:\{1,\dots,|H_m(x)|\}\to H_m(x) \label{eq:si:bijection}
}
such that given $(x,m,i)$ one can compute $y=\pi_{x,m}(i)$. Hence
\cgather{
\K(y\mid x,m)\le \log|H_m(x)|+O(1) \label{eq:si:condKup}
}
for all $y\in H_m(x)$. For the lower tail, define
\cgather{
A_\Delta=\{y\in H_m(x):\K(y\mid x,m)\le \log|H_m(x)|-\Delta\}. \label{eq:si:Adef}
}
Each $y\in A_\Delta$ has a prefix-free description of length at most
$\log|H_m(x)|-\Delta$, so $|A_\Delta|\le |H_m(x)|2^{-\Delta+O(1)}$. Thus
\cgather{
\Pr[y\in A_\Delta]\le 2^{-\Delta+O(1)}. \label{eq:si:Atail}
}
Since $x$ is incompressible, the two-part code through $H_m(x)$ is optimal up to $O(\log n)$ terms. Therefore
\cgather{
\Delta_y(H_m(x))
=
\log|H_m(x)|-\K(y\mid x,m)\pm O(\log n), \label{eq:si:deficiencyexpand}
}
which yields the stated asymptotic bound.
\end{proof}

\begin{corollary}[Expected discovery under binomial mutation]\label{cor:si:Phi}
Let $X\sim \mathrm{Unif}([q]^n)$ be the parent sequence.
Let $M\sim \mathrm{Binomial}(n,\mu)$ denote the number of mutated sites,
and conditional on $(X,M=m)$ let $Y\sim \mathrm{Unif}(H_m(X))$. Define the expected discovery probability
\mltlne{
\Phi(\mu)
:=
\mathbb{P}\left(\Delta_Y(H_M(X)) \ge \Delta \right)
\\=
\sum_{m=1}^n
\Pr(M=m)\,
\Pr_{y\sim H_m(X)} \left[\Delta_y(H_m(X)) \ge \Delta \right]. \label{eq:si:PhiDef}
}
Then, for $\Delta=o(\log n)$,
\cgather{
\Phi(\mu)\asymp
2^{-\Delta}
\sum_{m=1}^n
\Pr(M=m)
\frac{1}{|H_m(X)|}. \label{eq:si:PhiAsymp}
}
\end{corollary}

\begin{proof}
For $X\sim \mathrm{Unif}([q]^n)$, the standard incompressibility bound
\cgather{
\Pr\left[K(X)\le n\log q - t\right]\le q^{-t} \label{eq:si:incomp}
}
implies that $X$ satisfies the incompressibility condition of Lemma~\ref{lem:si:tight}
with overwhelming probability. On this event, Lemma~\ref{lem:si:tight} yields
\cgather{
\Pr_{y\sim H_m(X)}
\left[\Delta_y(H_m(X)) \ge \Delta \right]
\asymp
\frac{2^{-\Delta}}{|H_m(X)|}. \label{eq:si:probm}
}
The contribution of the exceptional set $\{X: K(X)\le n\log q - t\}$ is at most $q^{-t}$ and is negligible
for $t=\omega(1)$. Substituting the tight bound into the definition of $\Phi(\mu)$ gives the stated expression.
\end{proof}

Note that upon substituting the binomial mutation model and the sphere cardinality
$|H_m(X)|=\binom{n}{m}(q-1)^m$, the combinatorial factor $\binom{n}{m}$ appearing in $\Pr(M=m)$ cancels
with the identical factor in $|H_m(X)|$. Thus the multiplicity of mutation-location choices does not amplify
discovery probability: it is already accounted for by the mutation process itself. What remains is the symbol-choice
entropy $(q-1)^m$ and the exponential suppression arising from the rarity of exceptional strings within each
mutation-accessible model. We can therefore state the small-$\mu$ asymptotics of the discovery rate as follows.

\begin{lemma}[Small-$\mu$ asymptotics of discovery rate]\label{lem:si:smallmu}
For $\mu=o(1)$,
\cgather{
\sum_{m=1}^n\left(\frac{\mu}{q-1}\right)^m
=
\frac{\mu}{q-1}+O(\mu^2),
\qquad
(1-\mu)^n=e^{-n\mu+O(n\mu^2)}. \label{eq:si:smallmu}
}
Consequently,
\cgather{
\Phi(\mu)\asymp
C\,2^{-\Delta}\,\frac{\mu}{q-1}\,e^{-n\mu+O(n\mu^2)}, \label{eq:si:PhiSmallmu}
}
where $C$ absorbs polylogarithmic factors.
\end{lemma}

\begin{proof}
The first expansion is the truncated geometric series. Substituting
$n\log(1-\mu)=-n\mu+O(n\mu^2)$ into the expression for $\Phi(\mu)$ gives the claim.
\end{proof}

\begin{theorem}[Optimal mutation rate]\label{thm:si:opt}
In the regime $\mu=o(1)$ with $n\mu^2=o(1)$, the leading term $\mu e^{-n\mu}$ is maximized at $n\mu=1$.
Consequently,
\cgather{
\mu^\star=\frac{1+o(1)}{n}. \label{eq:si:muopt}
}
\end{theorem}

\begin{corollary}[Drake’s rule]\label{cor:si:drake}
Optimizing information-theoretic structure discovery under random mutation yields an inverse scaling of per-site
mutation rate with genome length.
\end{corollary}
