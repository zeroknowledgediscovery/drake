\PassOptionsToPackage{usenames,x11names, dvipsnames, svgnames}{xcolor}
\documentclass[9pt,twocolumn]{IEEEtran}
% Use the lineno option to display guide line numbers if required.
\usepackage{orcidlink}   % <— add this
% \input{preamble_ieee.tex}
\usepackage{tikz}
\usepackage{pgfplots}
\pgfplotsset{compat=1.18}
%\usepackage{pdfpages}
%\usepackage{shellesc}
%\ShellEscape{pdflatex si}
\usetikzlibrary{shapes,calc,shadows,fadings,arrows,decorations.pathreplacing,automata,positioning}
%\usepackage{hyperref}
\usetikzlibrary{external}
\tikzexternalize[prefix=./Figures/External/]% activate
\tikzexternaldisable 
\usetikzlibrary{decorations.text}
\usepgfplotslibrary{colorbrewer} 
\usepgfplotslibrary{statistics}
\usepgfplotslibrary{fillbetween}
%\usepackage[linesnumbered,ruled,vlined]{algorithm2e}
\usepackage[algo2e,ruled,vlined]{algorithm2e}

\usepackage{mathtools}
\usepackage{amssymb,amsfonts,amsmath,amsthm}
\newtheorem{theorem}{Theorem}
\newtheorem{lemma}{Lemma}
\newtheorem{corollary}{Corollary}
\newtheorem{definition}{Definition}

\newtheorem{exmpl}{Example}
\newtheorem{rem}{Remark}
\newtheorem{notn}{Notation}
\usepackage{xspace}
\newcommand{\abs}[1]{\left\lvert #1 \right\rvert}%

\newcommand{\set}[1]{\left\{ #1 \right\}}
\newcommand{\paren}[1]{\left( #1 \right)}
\newcommand{\bracket}[1]{\left[ #1 \right]}
% \newcommand{\norm}[1]{\left\Vert #1 \right\Vert}
\newcommand{\nrm}[1]{\left\llbracket{#1}\right\rrbracket}
\newcommand{\parenBar}[2]{\paren{#1\,{\left\Vert\,#2\right.}}}
\newcommand{\parenBarl}[2]{\paren{\left.#1\,\right\Vert\,#2}}
\newcommand{\ie}{$i.e.$\xspace}
\newcommand{\addcitation}{\textcolor{black!50!red}{\textbf{ADD CITATION}}}
\newcommand{\subtochange}[1]{{\color{black!50!green}{#1}}}
\newcommand{\tobecompleted}{{\color{black!50!red}TO BE COMPLETED.}}

\DeclareMathOperator*{\argmax}{argmax}
\DeclareMathOperator*{\argmin}{arg\,min}
\DeclareMathOperator*{\expect}{\mathbf{E}}
\DeclareMathOperator*{\var}{\mathbf{Var}}

\newcommand{\pIn}{\mathscr{P}_{\textrm{in}}}
\newcommand{\pOut}{\mathscr{P}_{\textrm{out}}}
\newcommand{\aIn}[1][\Sigma]{#1_{\textrm{in}}}
\newcommand{\aOut}[1][\Sigma]{#1_{\textrm{out}}}
\newcommand{\xin}[1]{#1_{\textrm{in}}}
\newcommand{\xout}[1]{#1_{\textrm{out}}}

\newcommand{\R}{\mathbb{R}} % Set of real numbers
\newcommand{\F}[1][]{\mathcal{F}_{#1}}
\newcommand{\SR}{\mathcal{S}} % Semiring of sets
\newcommand{\RR}{\mathcal{R}} % Ring of sets
\newcommand{\N}{\mathbb{N}} % Set of natural numbers (0 included)

\newcommand{\pitilde}{\widetilde{\pi}}
\newcommand{\Pitilde}{\widetilde{\Pi}}

\newcommand{\Pp}[1][n]{\mathscr{P}^+_{#1}}
%% ## Equation Space Control---------------------------
\def\EQSP{2pt}
\newcommand{\mltlne}[2][\EQSP]{\begingroup\setlength\abovedisplayskip{#1}\setlength\belowdisplayskip{#1}\begin{equation}\begin{multlined} #2 \end{multlined}\end{equation}\endgroup\noindent}
\newcommand{\cgather}[2][\EQSP]{\begingroup\setlength\abovedisplayskip{#1}\setlength\belowdisplayskip{#1}\begin{gather} #2 \end{gather}\endgroup\noindent}
\newcommand{\cgathers}[2][\EQSP]{\begingroup\setlength\abovedisplayskip{#1}\setlength\belowdisplayskip{#1}\begin{gather*} #2 \end{gather*}\endgroup\noindent}
\newcommand{\calign}[2][\EQSP]{\begingroup\setlength\abovedisplayskip{#1}\setlength\belowdisplayskip{#1}\begin{align} #2 \end{align}\endgroup\noindent}
\newcommand{\caligns}[2][\EQSP]{\begingroup\setlength\abovedisplayskip{#1}\setlength\belowdisplayskip{#1}\begin{align*} #2 \end{align*}\endgroup\noindent}
\newcommand{\mnp}[2]{\begin{minipage}{#1}#2\end{minipage}} 
\newif\iftikzX
\tikzXtrue
\tikzXfalse 
\newif\ifFIGS  
\FIGSfalse  
\FIGStrue

\def\EXTENDEDDATA{Extended Data\xspace}
\def\SUPPLEMENTARY{Supplementary\xspace}
\def\Methods{Methods\xspace}
\newcounter{Dcounter}
\setcounter{Dcounter}{1}
\newcommand{\DQS}[1]{
    \marginpar{
      \tikzexternaldisable 
      \tikz{
        \node[
          rounded corners=5pt,
          draw=none,
          thick,
          fill=black!10,
          font=\sffamily\fontsize{7}{8}\selectfont
        ]{\mnp{.3in} {\color{Red3}\raggedright  \#\theDcounter.~#1}}; 
      }
    }
  \stepcounter{Dcounter}\xspace
}

\newcommand{\K}{K}
\newcommand{\Hamming}{\mathrm{H}}
\newcommand{\SIapp}{SI Appendix}

\pgfplotsset{set layers=standard}
\pgfplotsset{compat=1.18} % or your version
   \pgfplotsset{
        % define the layers you need.
        % (Don't forget to add `main' somewhere in that list!!)
        layers/my layer set/.define layer set={
            background,
            main,
            foreground
        }{
            % you could state styles here which should be moved to
            % corresponding layers, but that is not necessary here.
            % That is why we don't state anything here
        },
        % activate the newly created layer set
        set layers=my layer set,
}
\begin{document}  



\begin{figure}
  \tikzexternalenable 
  \tikzsetnextfilename{plotfig}
\centering
%\tikzXfalse 
\tikzXtrue
  \iftikzX  

   \pgfplotsset{
    discard if/.style 2 args={
        x filter/.append code={
            \edef\tempa{\thisrow{#1}}
            \edef\tempb{#2}
            \ifx\tempa\tempb
                \def\pgfmathresult{inf}
            \fi
        }
    },
    discard if not/.style 2 args={
        x filter/.append code={
            \edef\tempa{\thisrow{#1}}
            \edef\tempb{#2}
            \ifx\tempa\tempb
            \else
                \def\pgfmathresult{inf}
            \fi
        }
    }
}

\begin{tikzpicture}[font=\bf\sffamily\fontsize{9}{9}\selectfont]
  \def\COLA{black!80}
  \def\COLB{Red4}
  \def\COLC{MediumBlue}
  \def\COLD{Green4}
 
  \def\WDT{1.35in}
  \def\HGT{1.35in}
  \def\DATA{Figures/genomecompress.csv}
  \def\TEMP{Figures/temperature_scaling_sim_data.csv}

  
 \node[anchor=north west] (A) at (0,0) { \begin{tikzpicture}[anchor=center]\begin{axis}[black, legend columns=2,
      legend style={anchor=west,at={(-.25,1.25)},inner sep=3pt,draw=none,fill=white,fill opacity=.850,align=right,text opacity=1,font=\bf\sffamily\fontsize{7}{9}\selectfont},
      axis line style={lightgray, opacity=0.01, thin},%
        enlargelimits=true, 
        axis on top=true,
        width=\WDT,
        height=\HGT,
        scale only axis=true,
            tick align=outside,
    grid=both,
    grid style={dashed,semithick,draw=black, opacity=.25},
    xlabel style={yshift=-.010in},
    major tick length=0pt,
xticklabel style={
    /pgf/number format/.cd,
    fixed,
    precision=0,
    1000 sep={}
},
yticklabel style={
    /pgf/number format/.cd,
    fixed,
    precision=3
},xlabel={year},ylabel={$\langle \textrm{bits/base} \rangle$},
  xmin=2019.5    ]


% =========================
% H3N2
% =========================
\addplot[smooth,
    name path=H3N2upper,
    draw=none,
    forget plot
]
table[
    col sep=comma,
    x=year,
    y expr=\thisrow{ci_high}*1.002,
    discard if not={organism}{H3N2}
] {\DATA};

\addplot[smooth,
    name path=H3N2lower,
    draw=none,
    forget plot
]
table[
    col sep=comma,
    x=year,
    y expr=\thisrow{ci_low}*0.998,
    discard if not={organism}{H3N2}
] {\DATA};

\addplot[
    \COLA,
    fill=\COLA,
    fill opacity=0.15,
    draw=none,
    forget plot
] fill between[of=H3N2upper and H3N2lower];

\addplot[
    \COLA,
    smooth,
    thick,
    mark=*,
    mark options={scale=.9,fill=white}
]
table[
    col sep=comma,
    x=year,
    y=avg_bits_per_base_corrected,
    discard if not={organism}{H3N2}
] {\DATA};
\addlegendentry{H3N2 Influenza A}



% =========================
% E. coli
% =========================
\addplot[smooth,
    name path=ECOLIupper,
    draw=none,
    forget plot
]
table[
    col sep=comma,
    x=year,
    y expr=\thisrow{ci_high}*1.001,
    discard if not={organism}{ecoli}
] {\DATA};

\addplot[smooth,
    name path=ECOLIlower,
    draw=none,
    forget plot
]
table[
    col sep=comma,
    x=year,
    y expr=\thisrow{ci_low}*0.999,
    discard if not={organism}{ecoli}
] {\DATA};

\addplot[
    \COLC,
    fill=\COLC,
    fill opacity=0.15,
    draw=none,
    forget plot
] fill between[of=ECOLIupper and ECOLIlower];

\addplot[
    \COLC,
    smooth,
    thick,
    mark=*,
    mark options={scale=.7,fill=\COLC}
]
table[
    col sep=comma,
    x=year,
    y=avg_bits_per_base_corrected,
    discard if not={organism}{ecoli}
] {\DATA};
\addlegendentry{E.\ coli}


% =========================
% SARS-CoV-2
% =========================
\addplot[smooth,
    name path=COVupper,
    draw=none,
    forget plot
]
table[
    col sep=comma,
    x=year,
    y expr=\thisrow{ci_high}*1,
    discard if not={organism}{hCOV-19}
] {\DATA};

\addplot[smooth,
    name path=COVlower,
    draw=none,
    forget plot
]
table[
    col sep=comma,
    x=year,
    y expr=\thisrow{ci_low}*1,
    discard if not={organism}{hCOV-19}
] {\DATA};

\addplot[
    \COLB,
    fill=\COLB,
    fill opacity=0.15,
    draw=none,
    forget plot
] fill between[of=COVupper and COVlower];

\addplot[
    \COLB,
    smooth,
    thick,
    mark=*,
    mark options={scale=.9,fill=white}
]
table[
    col sep=comma,
    x=year,
    y=avg_bits_per_base_corrected,
    discard if not={organism}{hCOV-19}
] {\DATA};
\addlegendentry{SARS-CoV-2}

% =========================
% Mpox
% =========================
\addplot[smooth,
    name path=POXupper,
    draw=none,
    forget plot
]
table[
    col sep=comma,
    x=year,
    y expr=\thisrow{ci_high}*1.0020,
    discard if not={organism}{Pox}
] {\DATA};

\addplot[smooth,
    name path=POXlower,
    draw=none,
    forget plot
]
table[
    col sep=comma,
    x=year,
    y expr=\thisrow{ci_low}*0.998,
    discard if not={organism}{Pox}
] {\DATA};

\addplot[
    \COLD,
    fill=\COLD,
    fill opacity=0.15,
    draw=none,
    forget plot
] fill between[of=POXupper and POXlower];

\addplot[
    \COLD,
    smooth,
    thick,
    mark=square*,
    mark options={scale=.7,fill=\COLD}
]
table[
    col sep=comma,
    x=year,
    y=avg_bits_per_base_corrected,
    discard if not={organism}{Pox}
] {\DATA};
\addlegendentry{Mpox}
       

 
  \end{axis}
\end{tikzpicture}
 };


       \def\OPC{.5}
       \def\SCL{.6}
\def\COLA{black!50}
  \def\COLB{Green2}
  \def\COLC{MediumBlue}
  \def\COLD{Green4}
\def\HGTA{1.3in}

       \node[anchor=south west] (B) at ([xshift=.2in]A.south east) { \begin{tikzpicture}[anchor=center]


           \begin{axis}[black, legend columns=2,
      legend style={anchor=west,at={(-.37,1.22)},inner sep=3pt,draw=none,fill=white,fill opacity=.850,align=right,text opacity=1,font=\bf\sffamily\fontsize{7}{9}\selectfont},
      axis line style={lightgray, opacity=0.01, thin},%
        enlargelimits=true, 
        axis on top=true,
        width=\WDT,
        height=\HGTA,
        scale only axis=true,
            tick align=outside,
    grid=major,
    grid style={dashed,semithick,draw=black, opacity=.25},
    xlabel style={yshift=-.010in},
    major tick length=0pt,
     minor tick length=0pt,
   xmode=log,ymode=log,
   xlabel={mutation rate  $\mu$},ylabel={discovery rate},
   ymin=1e-50
     ]


\addplot[\COLA, smooth, thick, mark=none,mark options={scale=.4,fill=white}] table [ col sep=comma,
        y=phi_theory,
        x expr=\thisrow{mu}, discard if not={n}{1000}
] {\TEMP};
\addlegendentry{Th. $n=1000$}

\addplot[\COLA, smooth, only marks, , mark=*,mark options={scale=\SCL,,opacity=\OPC}] table [ col sep=comma,
       y=phi_mc,
       x expr=\thisrow{mu}, discard if not={n}{1000}
] {\TEMP};
\addlegendentry{mc. $n=1000$}

\addplot[\COLB, smooth, thick, mark=none,mark options={scale=.4,fill=white}] table [ col sep=comma,
        y=phi_theory,
        x expr=\thisrow{mu}, discard if not={n}{10000}
        ] {\TEMP};
\addlegendentry{Th. $n=10,000$}
\addplot[\COLB, smooth, , only marks, mark=*,mark options={scale=\SCL,,opacity=\OPC}] table [ col sep=comma,
        y=phi_mc,
        x expr=\thisrow{mu}, discard if not={n}{10000}
        ] {\TEMP};
\addlegendentry{mc. $n=10,000$}



\addplot[\COLC, smooth, thick, mark=none,mark options={scale=.4,fill=white}] table [ col sep=comma,
        y=phi_theory,
        x expr=\thisrow{mu}, discard if not={n}{100000}
        ] {\TEMP};
\addlegendentry{Th. $n=100,000$}

\addplot[\COLC, smooth, , only marks, mark=*,mark options={scale=\SCL,,opacity=\OPC}] table [ col sep=comma,
        y=phi_mc,
        x expr=\thisrow{mu}, discard if not={n}{100000}
        ] {\TEMP};
\addlegendentry{mc. $n=100,000$}


 
  \end{axis}
\end{tikzpicture}
 };






\def\GuideScale{3e-07}
\def\OPCA{1}

 \node[anchor=north west] (C) at ([yshift=-.3in]B.south west) { \begin{tikzpicture}[anchor=center]\begin{axis}[black, legend columns=2,
      legend style={anchor=west,at={(.2,1.25)},inner sep=3pt,draw=none,fill=white,fill opacity=.850,align=right,text opacity=1,font=\bf\sffamily\fontsize{9}{9}\selectfont},
      axis line style={lightgray, opacity=0.01, thin},%
        enlargelimits=true, 
        axis on top=false,
        width=\WDT,
        height=\HGT,
        scale only axis=true,
            tick align=outside,
    grid=major,
    grid style={dashed,semithick,draw=black, opacity=.250},
    xlabel style={yshift=-.040in,xshift=-.2in},
    major tick length=0pt,
     minor tick length=0pt,
   xmode=log,ymode=log,   ymin=1e-50,
xlabel={mutation temperature $T=n\mu$},ylabel={discovery rate}, use fpu=true,  /pgf/fpu/output format=fixed,
   ]


\addplot[forget plot,\COLA, smooth, ,only marks, ,opacity=\OPC,  mark=*,mark options={scale=\SCL,opacity=\OPC}] table [ col sep=comma,
        y=phi_mc,
        x expr=\thisrow{mu}*1000, discard if not={n}{1000}
] {\TEMP};

\addplot[forget plot,\COLB, smooth, , only marks,,opacity=\OPC, mark=*,mark options={scale=\SCL,,opacity=\OPC}] table [ col sep=comma,
        y=phi_mc,
        x expr=\thisrow{mu}*10000, discard if not={n}{10000}
] {\TEMP};

\addplot[forget plot,\COLC, smooth, , only marks,,opacity=\OPC, mark=*,mark options={scale=\SCL,,opacity=\OPC}] table [ col sep=comma,
        y=phi_mc,
        x expr=\thisrow{mu}*100000, discard if not={n}{100000}
        ] {\TEMP};




       
\addplot[forget plot,\COLA, smooth, thick,,opacity=\OPCA, mark=none,mark options={scale=.4,fill=white,opacity=\OPC}] table [ col sep=comma,
        y=phi_theory,
        x expr=\thisrow{mu}*1000, discard if not={n}{1000}
] {\TEMP};



\addplot[forget plot,\COLB, smooth, thick,,opacity=\OPCA, mark=none,mark options={scale=.4,fill=white,opacity=\OPC}] table [ col sep=comma,
        y=phi_theory,
        x expr=\thisrow{mu}*10000, discard if not={n}{10000}
] {\TEMP};



\addplot[forget plot,\COLC, smooth, thick, ,opacity=\OPCA,mark=none,mark options={scale=.4,fill=white,opacity=\OPC}] table [ col sep=comma,
        y=phi_theory,
        x expr=\thisrow{mu}*100000, discard if not={n}{100000}
        ] {\TEMP};


\addplot[
    Red1,
    ultra thick,    domain=1e-2:1e2,    
unbounded coords=discard,
    samples=200,                on layer=foreground,
]
{ \GuideScale *x*exp(-x) };
\addlegendentry{$T e^{-T}$}
 


  \end{axis}
\end{tikzpicture}
 };


       
\def\HGTB{1.32in}
\def\TOY{Figures/figS1_toy_entropy_proxy_plotdata.csv}
\def\COLX{black}
\def\COLY{DarkOrange2}
\node[anchor=north west] (D) at ($(C.north west)!(A.west)!(C.north)$) {
\begin{tikzpicture}[anchor=center]\begin{axis}[black, legend columns=1,
      legend style={anchor=west,at={(-.2,1.3)},inner sep=3pt,draw=none,fill=white,fill opacity=.850,align=right,text opacity=1,font=\bf\sffamily\fontsize{9}{9}\selectfont},
      axis line style={lightgray, opacity=0.01, thin},%
        enlargelimits=true, 
        axis on top=false,
        width=\WDT,
        height=\HGTB,
        scale only axis=true,
            tick align=outside,
    grid=major,
    grid style={dashed,semithick,draw=black, opacity=.250},
    xlabel style={yshift=.020in,xshift=-.2in},
    major tick length=0pt,
     minor tick length=0pt,  set layers=standard,
     %xmode=log,ymode=log,   ymin=1e-50,
     ylabel style={align=center},
xlabel={generation},ylabel={Bigram conditional entropy\\ $H(X_t\mid X_{t-1})$ (bits/symbol)}, use fpu=true,  /pgf/fpu/output format=fixed,
   ]


\addplot[, very thick, dashed, smooth,\COLX, opacity=1,, mark=none ] table [ col sep=comma,
        y=mut_mean,
        x=generation, %% discard if not={n}{1000}
] {\TOY};
\addlegendentry{Mutation only}

\addplot[smooth,
    name path=mutlower,
    draw=none,
    forget plot
]
table[
    col sep=comma,
    x=generation,
    y expr=\thisrow{mut_ci_low}*0.99,
] {\TOY};
\addplot[smooth,
    name path=mutupper,
    draw=none,
    forget plot
]
table[
    col sep=comma,
    x=generation,
    y expr=\thisrow{mut_ci_high}*1.02,
] {\TOY};

\addplot[
    \COLX,
    fill=\COLX,
    fill opacity=0.35,
    draw=none,
    forget plot
] fill between[of=mutupper and mutlower];


\addplot[, very thick, dashed, smooth,\COLY, smooth, mark=none ] table [ col sep=comma,
        y=sel_mean,
        x=generation, %% discard if not={n}{1000}
] {\TOY};
\addlegendentry{Mutation + selective filter}

\addplot[smooth,
    name path=mutsellower,
    draw=none,
    forget plot
]
table[
    col sep=comma,
    x=generation,
    y expr=\thisrow{sel_ci_low}*0.99,
] {\TOY};
\addplot[smooth,
    name path=mutselupper,
    draw=none,
    forget plot
]
table[
    col sep=comma,
    x=generation,
    y expr=\thisrow{sel_ci_high}*1.02,
] {\TOY};

\addplot[
    \COLD,
    fill=\COLY,
    fill opacity=0.5,
    draw=none,
    forget plot
] fill between[of=mutselupper and mutsellower];


  \end{axis}
\end{tikzpicture}

  };





  \node[anchor=south west] (L1) at (A.north west) {{\Large a.}};
  \node[anchor=south west] (L2) at (B.north west) {{\Large c.}};
  \node[anchor=south west] (L3) at (C.north west) {{\Large d.}};
  \node[anchor=south west] (L4) at (D.north west) {{\Large b.}};
\end{tikzpicture}

\else
  \includegraphics[width=.35\textwidth]{Figures/External/plotfig}
\fi
\end{figure}
 
\end{document}
